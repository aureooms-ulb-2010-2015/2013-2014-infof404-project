\section{About}

Least Slack Time (LST) scheduling is a scheduling algorithm. It assigns priority based on the slack time of a process. Slack time is the amount of time left after a job if the job was started now. This algorithm is also known as Least Laxity First. Its most common use is in embedded systems, especially those with multiple processors. It imposes the simple constraint that each process on each available processor possesses the same run time, and that individual processes do not have an affinity to a certain processor. This is what lends it a suitability to embedded systems.