\subsection{Colors}
You can highlight the prime and composite numbers in any color you want using the options

\begin{itemize}
	\item{-{}-prime-color r g b}
	\item{-{}-composite-color r g b}
\end{itemize}

\subsection{PPM filters}
The prime numbers are highlighted by applying ppm filters on them.

\begin{itemize}
	\item{-{}-prime-filter <ppm src>}
	\item{-{}-composite-filter <ppm src>}
\end{itemize}

The pixels used will be a centered square of the image with side computed as $min(height, width)$.

This parameter allows to apply any color transformation on the pixels since the source image could be anything.

\begin{figure}
	\centering
	\includegraphics[width=0.8\textwidth]{eps/filter}
	\caption{\label{fig:repr:1} An example of filter applied to the Ulam Spiral}
\end{figure}


\subsection{Steganography}
Using the PPM filters one can manage to hide a picture inside an other one. This is not very obscure (best when -{}-composite-filter is noise), but the algorithm allows to only retrieve the pixels that are \emph{prime} retrieving the information that was hidden due to the low density of primes.


\begin{figure}
	\centering
	\includegraphics[width=0.8\textwidth]{eps/steganography}
	\caption{\label{fig:repr:2} \ref{fig:repr:1} hidden in an other image}
\end{figure}
