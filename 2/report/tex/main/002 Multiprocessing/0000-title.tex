\newpage\cleardoublepage\phantomsection
\section{A multiprocessor system}


We have used the Message Passing Interface to generate prime numbers with several processors. We can show that appart from the communication overhead, the work has been split equally between processes. \\ Each processor is assigned a range of number for which he has to find the primes. A processor can be in two \"states\":
\begin{itemize}
	\item{finding primes: }
	Only one processor at the time can be finding primes. This processor go through his range of numbers and identifies the primes. When a prime is found it sends a message to processors taking care of ranges higher than his. The message sent contains required information to discard numbers in higher ranges. 
	\item{receiving primes: }
	Those processors are waiting for messages. When they receive one they go through their range of numbers and discard the non-prime numbers. 

\end{itemize}


A message is sent like in \ref{code:prime:1} with:

\begin{itemize}
	\item{$mpi\_rank$ :}
	Value used to determine the sending processor's state
	
	\item{$i,l, left|right$ :}
	Value used to compute the index of a number in the receiver's processor range
	\item{$k$ :} 
	The prime found for which multiple need to be discarded.
	\item{$forward[mpi\_rank]$ :}
	This processor's forward table, containing all processors taking care of range higher than his.

\end{itemize}

\myinputminted[firstline=93,lastline=97]{c++}{../h/os/alg.hpp}{sending prime information}{code:prime:1}{1}

Two main difficulties have occured while designing the multiprocessor architechture.
