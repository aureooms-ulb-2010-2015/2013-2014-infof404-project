\subsection{Writing the files}
When the processors finish the last step is to write in the .ppm file. We implemented two differents strategies.

\subsubsection{Random strategy}
When a processor finishes he directly writes in the ppm file. This method is not very efficient because it takes a whole disk block, flips one bit and writes it all back on the disk. This method uses $$MPI\_Reduce(\&tmp, \&prime\_n, 1, MPI\_SIZE\_T, MPI\_SUM, 0, MPI\_COMM\_WORLD);$$  

\begin{itemize}

	\item{$tmp$ :}
	\item{$MPI_SIZE_T$ :}
	\item{$MPI_SUM$ :} 

\end{itemize}

\subsubsection{Sequencial strategy}
When all processors are finished the pixels are written one by one. The block size is mesured and one pixel is written on three block. each composant of the pixel is on one block. We fill a buffer $byte buffer[3*block\_size]$ and flush it into the file for every pixel.

\subsection{Remark}
We could have allocated the range of numbers to the processors depending on the strucutre of the ulam spiral. Each processor would then take care of one line of the file instead of one range in natural numbers. This would make the writing of the ppm file easier but would have made the communication between processors much more complex.