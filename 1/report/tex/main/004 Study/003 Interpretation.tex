\subsection{Interpretation}

The results are not surprising.

\begin{conditions}
	$S$ & schedulability rate\\
	$P$ & preemption rate
\end{conditions}

\begin{equation}
	\uparrow U \Rightarrow ~\downarrow S ~~\land ~\uparrow P
	\label{eq:U influence}
\end{equation}

\begin{equation}
	\uparrow \Delta_r \Rightarrow ~\downarrow S ~~\land ~\downarrow P
	\label{eq:d influence}
\end{equation}

\begin{equation}
	\uparrow n \Rightarrow ~\downarrow S ~~\land ~\uparrow P
	\label{eq:n influence}
\end{equation}


The exponential decrease of the schedulability rate in $u$ shown in \ref{fig:stu:au} suggests that
to achieve a good use of the whole computation capabilities of the cpu some engineering to design the task system has to be done.