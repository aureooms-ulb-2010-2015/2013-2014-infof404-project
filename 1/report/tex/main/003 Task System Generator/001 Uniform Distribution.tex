\subsection{Uniform distribution}

\subsubsection{Random partitioning}

We focused on producing uniformely distributed utilizations.

Suppose we want to generate a system $\tau = \{\tau_1 , \tau_2 , \dots , \tau_n \}$ with

\begin{equation}
	C_i \geq 1 \qquad \forall 1 \leq i \leq n
	\label{eq:WCET minimum}
\end{equation}

\begin{equation}
	\sum_{i=1}^{n} U_i + \epsilon = U
	\label{eq:Usage shift}
\end{equation}

\begin{equation}
	\text{minimize}~\epsilon
	\label{eq:Economic function of the generator}
\end{equation}

\begin{equation}
	\epsilon \geq 0
	\label{eq:Usage shift polarity}
\end{equation}

We choose $n-1$ partition $p_j \in P = ~]0, U[$ with

\begin{equation}
	|p_j - p_k| \geq \frac{1}{T_{min}} \qquad \forall p_j \in P, p_k \in P \cup \{0, U\}, p_j \neq p_k
	\label{eq:Discrete partition condition}
\end{equation}

To achieve \ref{eq:Discrete partition condition} we define the \emph{floor\_min\_uniform} function (see \ref{code:gen:1}).

\myinputminted[firstline=10,lastline=17]{c++}{../h/os/generator.h}{The \emph{floor\_min\_uniform} function}{code:gen:1}{1}

An example of use can be seen in \ref{code:app:4}.

(\href{../h/os/generator.h}{Link to the source code}.)

\subsubsection{Limits}

The \emph{lowest common multiple} is exponential in $n$. If we only use the primitive types provided by the \CXX~language we are bounded to a \emph{max} value.

Sufficient condition for the feasability of lcm computation

\begin{equation}
	T_{max}^n \leq 2^b-1
	\label{eq:Lowest common multiple condition}
\end{equation}

Where
\begin{conditions}
	T_{max}		&	the maximum value of the period distribution\\
	n			&	number of tasks in the system \\
	2^b-1		&	the maximum value for an integer
\end{conditions}

Another requirement was to never overflow the $U$ asked by the user.

Sufficient condition for a non-overflowing total usage

\begin{equation}
	U_i \geq \frac{1}{T_i}
	\label{eq:Usage no-overflow warranty}
\end{equation}

Where
\begin{conditions}
	U_i	&	usage of $\tau_i$ \\
	T_i	&	period of $\tau_i$
\end{conditions}

For \ref{eq:Economic function of the generator} we can note that

\begin{equation}
	\epsilon_{max} = \frac{n}{T_{max}}
	\label{eq:Usage shift maximum}
\end{equation}

We decided not to attach much importance to \ref{eq:Economic function of the generator} because of the spectrum of study considered (see \ref{sec:study}).